\section{Lessons Learned Perspective} \label{sec:lessons}

During the development of the project, we had some issues while setting up new services in Docker. We did not have prior experience with many of the services such as Grafana, Prometheus, and Kibana, and therefore we found that it took way longer than expected to get these services to run properly. There was not a simple solution, other than trying until we got it right, and this was partly because of how hard it was to narrow down problems as the Docker logs are not always that elaborate. From this we learned to dedicate more time to tasks that involve the implementation of new services.

Our project utilize many different tools for code quality, and security assessment, hereunder Snyk. We implemented Snyk into our CI/CD chain, where it's supposed to stop the deployment if it finds any major problems with the commit. This ran fine for a while until something suddenly went wrong in the setup. From this point on Snyk failed our CI/CD pipeline, not because of any vulnerabilities, but because of some other unidentified problems\footnote{Problem with Snyk can be seen in GitHub Actions: \href{https://github.com/Lindharden/DevOps/actions/runs/4660817522/jobs/8261515084?pr=98}{github.com/Lindharden/DevOps/actions/runs/4660817522}.}. We didn't find any solution to this, so we solved the problem by disabling Snyk entirely. We deemed this fine for now, as we have other tools which also check for vulnerabilities in our application.

\subsection{DevOps style of work}
For a majority of other school related development projects we have not utilized a concrete tool to organise tasks and issues. For this project we took great advantage of GitHub Issues to have a centralized backlog that was also streamlined using issue templates. GitHub Issues provided a much needed overview of features and bugs and made it easy to manage and distribute responsibilities.

The DevOps handbook \cite{kim2021devops} mentions the Three Ways. This is our thoughts on how we adhered to each of the principles:
\begin{itemize}
    \item \textbf{Flow}: We adhere to the Flow principle which talks about implementing a fast left-to-right flow, which is the time it takes from when requests are put on the backlog, till they are implemented and is in production. Our setup has a fast Flow as we have a CI/CD setup which automatically builds, tests and deploys changes from the main branch to our remote server at DigitalOcean. This means that the time it takes from when changes are committed, till they are in production, is just as long as it takes for our team to approve and merge pull-requests. We also try to split tasks into as small as possible backlog items (GitHub Issues), such that they are easier to delegate to different group members, and thus pass through the pipeline faster.
    \item \textbf{Feedback}: We adhere to the Feedback principle, which focuses on continuous problem solving when they occur. Our application don't have real users, but we simulate the feedback process by receiving error messages from the user simulation run by the teaching team. When we see that errors occur, we create an Issue/ticket which goes in our backlog. We then try to fix the error as fast as possible.
    \item \textbf{Continual Learning and Experimentation}: We try to adhere to Continual Learning and Experimentation where we encourage risk taking and see mistakes as opportunities to learn. We are in a special situation as we are not implementing a real application, and therefore we can afford to make big mistakes without any bigger impacts. We relieve the pressure of having to manually deploy the changes we make, by having an CI/CD setup which automatically deploys all the changes we make.
\end{itemize}
